\documentclass[xcolor=svgnames]{beamer} 
%\usecolortheme[named=PowderBlue]{structure} %structure
\usecolortheme{beaver} %structure
\useoutertheme{infolines} 
%\usetheme[height=7mm]{CambridgeUS} %Rochester
\usetheme{CambridgeUS} %Rochester
\setbeamertemplate{items}[ball] 
\setbeamertemplate{blocks}[rounded][shadow=true] 
\setbeamertemplate{navigation symbols}{} 

\usepackage{xeCJK}
\setCJKmainfont{Hiragino Sans GB}
\usepackage{tikz}
\usepackage{listings}
\usetikzlibrary{shapes,arrows}

\usepackage{color}
 
\definecolor{dkgreen}{rgb}{0,0.6,0}
\definecolor{gray}{rgb}{0.5,0.5,0.5}
\definecolor{mauve}{rgb}{0.58,0,0.82}
 
\lstset{ %
  %language=C,                % the language of the code
  basicstyle=\footnotesize,           % the size of the fonts that are used for the code
  %numbers=left,                   % where to put the line-numbers
  %numberstyle=\tiny\color{gray},  % the style that is used for the line-numbers
  %stepnumber=2,                   % the step between two line-numbers. If it's 1, each line 
                                  % will be numbered
  numbersep=5pt,                  % how far the line-numbers are from the code
  backgroundcolor=\color{white},      % choose the background color. You must add \usepackage{color}
  showspaces=false,               % show spaces adding particular underscores
  showstringspaces=false,         % underline spaces within strings
  showtabs=false,                 % show tabs within strings adding particular underscores
  frame=single,                   % adds a frame around the code
  rulecolor=\color{black},        % if not set, the frame-color may be changed on line-breaks within not-black text (e.g. commens (green here))
  tabsize=4,                      % sets default tabsize to 2 spaces
  captionpos=b,                   % sets the caption-position to bottom
  breaklines=true,                % sets automatic line breaking
  breakatwhitespace=false,        % sets if automatic breaks should only happen at whitespace
  title=\lstname,                   % show the filename of files included with \lstinputlisting;
                                  % also try caption instead of title
  keywordstyle=\color{blue},          % keyword style
  commentstyle=\color{dkgreen},       % comment style
  stringstyle=\color{mauve},         % string literal style
  escapeinside={\%*}{*)},            % if you want to add LaTeX within your code
  morekeywords={*,...}               % if you want to add more keywords to the set
}

\hypersetup{ pdfpagemode={FullScreen}}

\title{PYTHON NOTES}
\subtitle{with internals}
\author{Yang Bo}
\institute[Admaster]{
    RDBJ \\
    Admaster
}
\date{2015.5.25}

\begin{document}

\begin{frame}[plain]
  \titlepage
\end{frame}
%----------- slide -------------%
\begin{frame}{Outline of the talk} 
 
\begin{itemize} 
  \item Python Object
  \item Built-in Types 
  \item Technical lemmata 
  \item Proof of the main theorem 
  \item Conclusions 
\end{itemize} 
 
\end{frame} 

%------------ python object /pyobject--------%
\begin{frame}[fragile]{Python Object}
    \begin{lstlisting}[language=C]{object.h}
#define PyObject_HEAD                   \
    _PyObject_HEAD_EXTRA                \
    Py_ssize_t ob_refcnt;               \
    struct _typeobject *ob_type;

typedef struct _object {
    PyObject_HEAD
} PyObject;

#define PyObject_VAR_HEAD               \
    PyObject_HEAD                       \
    Py_ssize_t ob_size; /* Number of items in variable part */

typedef struct {
    PyObject_VAR_HEAD
} PyVarObject;
\end{lstlisting} 
\end{frame}

%----------- python object/ pyobject -------------%
\begin{frame}{Python Object}
  \begin{itemize}
    \item Foundations of Everything(interally)
    \item Not change size once created
    \item PyObject / PyVarObject
    \item Reference Count
    \item Type Info
    \item Size Info
    \item Layout
  \end{itemize}
\end{frame}


%------------ python object --------%
\begin{frame}[fragile]{Python Object}
    \begin{lstlisting}[language=C]{object.h}
#define _Py_Dealloc(op) (                               \
    _Py_INC_TPFREES(op) _Py_COUNT_ALLOCS_COMMA          \
    (*Py_TYPE(op)->tp_dealloc)((PyObject *)(op)))
#endif /* !Py_TRACE_REFS */

#define Py_INCREF(op) (                         \
    _Py_INC_REFTOTAL  _Py_REF_DEBUG_COMMA       \
    ((PyObject*)(op))->ob_refcnt++)

#define Py_DECREF(op)                                   \
    do {                                                \
        if (_Py_DEC_REFTOTAL  _Py_REF_DEBUG_COMMA       \
        --((PyObject*)(op))->ob_refcnt != 0)            \
            _Py_CHECK_REFCNT(op)                        \
        else                                            \
        _Py_Dealloc((PyObject *)(op));                  \
    } while (0)
\end{lstlisting} 
\end{frame}


%------------ python object / object and type--------%
\begin{frame}[fragile]{Python Object/object and type}
    <type 'object'> and <type 'type'>
    \begin{itemize}
        \item 'object' in python is PyBaseObject\_Type indeed, not PyObject
        \item 'object' is base class of everything, including 'type'
        \item type of all types is 'type'
        \item type of 'object' is 'type'
    \end{itemize}
    \begin{lstlisting}[language=C]{object.h}
PyAPI_DATA(PyTypeObject) PyType_Type; /* built-in 'type' */
PyAPI_DATA(PyTypeObject) PyBaseObject_Type; /* built-in 'object' */
    \end{lstlisting} 
\end{frame}

%------------ python object / example --------%
\begin{frame}[fragile]{Python Object/example}
    \begin{figure}
       \begin{center}
       \includegraphics[width=10cm, height=8cm]{pystring_hirechy.png}
       \end{center}
\end{figure}
\end{frame}

%------------ built-in types / int/ intobject--------%
\begin{frame}[fragile]{Built-in Types/Ineger}
\begin{lstlisting}[language=C]{intobject.h}
typedef struct {
    PyObject_HEAD
    long ob_ival;
} PyIntObject;
\end{lstlisting} 
\end{frame}

%------------ built-in types / int / general--------%
\begin{frame}{Built-in Types/Integer}
\begin{itemize} 
    \item cache small: [-5, 257)
  \item cache block: PyIntBlock, each 82
  \item block\_list and free\_list
  \item gc: int\_dealloc
  \item memory usage: largest quantity of integers at one exact time, not how many integers totally had in history
\end{itemize}
\end{frame}

%------------ built-in types / int / efficiency --------%
\begin{frame}[fragile]{Built-in Types/Integer}
\begin{lstlisting}[language=Python]
In [6]: %timeit float("1233")
The slowest run took 13.29 times longer than the fastest. This could mean that an intermediate result is being cached
1000000 loops, best of 3: 233 ns per loop

In [7]: %timeit int("1233")
The slowest run took 5.72 times longer than the fastest. This could mean that an intermediate result is being cached
1000000 loops, best of 3: 708 ns per loop
\end{lstlisting} 
Why?
Let's observe int\_new(objects/intobjects.c) and float\_new(objects/floatobjects.c).
\end{frame}

%------------ python object / string / stringobject--------%
\begin{frame}[fragile]{Built-in Types/String}
\begin{lstlisting}[language=C]{stringobject.h}
typedef struct {
    PyObject_VAR_HEAD
    long ob_shash; // init as -1
    int ob_sstate; // whether interned
    char ob_sval[1]; //trick, 'cos this is not only for gnu c
} PyStringObject;
\end{lstlisting} 
\end{frame}


%------------ python object /string / intern--------%
\begin{frame}{Built-in Types/String}
\begin{itemize} 
  \item not changable.
      \begin{itemize}
        \item the good: used as key in dict
        \item the bad: efficiency in join
      \end{itemize}
  \item you can have \textbackslash0 in your string, for size is determined by ob\_size and tp\_itemsize
  \item PyString\_FromString PyString\_FromStringAndSize
  \item intern
    \end{itemize} 
\end{frame}

%------------ python object / string / memory layout example--------%
\begin{frame}[fragile]{Built-in Types/String}
    \begin{figure}
       \begin{center}
       \includegraphics[width=10cm, height=3.5cm]{pystring_memory.png}
       \end{center}
\end{figure}
\end{frame}

%------------ python object /string--------%
\begin{frame}{Built-in Types/String}
intern
\begin{itemize} 
  \item not changable.
      \begin{itemize}
        \item the good: used as key in dict
        \item the bad: efficiency in join
      \end{itemize}
  \item you can have \textbackslash0 in your string, for size is determined by ob\_size and tp\_itemsize
  \item PyString\_FromString PyString\_FromStringAndSize
  \item intern
    \end{itemize} 
\end{frame}
\begin{frame}{流程图}
\tikzstyle{decision} = [diamond, draw, fill=blue!15, 
    text width=6em, text badly centered, node distance=3cm, inner sep=0pt]
\tikzstyle{block} = [rectangle, draw, fill=blue!15, 
    text width=5em, text centered, rounded corners, minimum height=3em]
\tikzstyle{line} = [draw, -latex']
\tikzstyle{cloud} = [draw, ellipse,fill=red!20, node distance=3cm,
    minimum height=2em]
    
\begin{tikzpicture}[node distance = 2cm, auto]
    % Place nodes
    \node [block] (init) {initialize model};
    \node [cloud, left of=init] (expert) {expert};
    \node [cloud, right of=init] (system) {system};
    \node [block, below of=init] (identify) {identify candidate models};
    \node [block, below of=identify] (evaluate) {evaluate candidate models};
    \node [block, left of=evaluate, node distance=3cm] (update) {update model};
    \node [decision, below of=evaluate] (decide) {is best candidate better?};
    \node [block, right of=decide, node distance=4cm] (stop) {stop};
    % Draw edges
    \path [line] (init) -- (identify);
    \path [line] (identify) -- (evaluate);
    \path [line] (evaluate) -- (decide);
    \path [line] (decide) -| node [near start] {yes} (update);
    \path [line] (update) |- (identify);
    \path [line] (decide) -- node {no}(stop);
    \path [line,dashed] (expert) -- (init);
    \path [line,dashed] (system) -- (init);
    \path [line,dashed] (system) |- (evaluate);
\end{tikzpicture}
\end{frame}

\begin{frame}{Splitting a slide into columns}

The line you are reading goes all the way across the slide.
From the left margin to the right margin.  Now we are going
to split the slide into two columns.
\bigskip

\begin{columns}
  \begin{column}{0.5\textwidth}
    Here is the first column.  We put an itemized list in it.
    \begin{itemize}
      \item This is an item
      \item This is another item
      \item Yet another item
    \end{itemize}
  \end{column}

  \begin{column}{0.3\textwidth}
    Here is the second column.  We will put a picture in it.
   
  \end{column}
\end{columns}
\bigskip
The line you are reading goes all the way across the slide.
From the left margin to the right margin.
\end{frame}
\begin{frame}[c]{} % [c] is the default
    \begin{center}\Huge
        Q \& A
    \end{center}
\end{frame} 
\end{document}
